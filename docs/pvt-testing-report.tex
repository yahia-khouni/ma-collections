\documentclass[12pt,a4paper]{article}
\usepackage[utf8]{inputenc}
\usepackage[T1]{fontenc}
\usepackage[french]{babel}
\usepackage{geometry}
\usepackage{graphicx}
\usepackage{hyperref}
\usepackage{booktabs}
\usepackage{longtable}
\usepackage{array}
\usepackage{xcolor}
\usepackage{colortbl}
\usepackage{titlesec}
\usepackage{fancyhdr}
\usepackage{enumitem}

\geometry{margin=2cm}

\definecolor{brandgold}{RGB}{212,175,55}
\definecolor{darkgrey}{RGB}{17,24,39}
\definecolor{passed}{RGB}{34,197,94}
\definecolor{failed}{RGB}{239,68,68}
\definecolor{pending}{RGB}{234,179,8}

\titleformat{\section}{\Large\bfseries\color{darkgrey}}{\thesection}{1em}{}
\titleformat{\subsection}{\large\bfseries\color{darkgrey}}{\thesubsection}{1em}{}

\pagestyle{fancy}
\fancyhf{}
\fancyhead[L]{\textcolor{brandgold}{M\&A Collections}}
\fancyhead[R]{Rapport de Tests PVT}
\fancyfoot[C]{\thepage}

\newcommand{\passed}{\cellcolor{passed!30}\textbf{PASSÉ}}
\newcommand{\failed}{\cellcolor{failed!30}\textbf{ÉCHOUÉ}}
\newcommand{\pending}{\cellcolor{pending!30}\textbf{EN COURS}}

\begin{document}

\begin{titlepage}
    \centering
    \vspace*{2cm}
    
    {\Huge\bfseries\textcolor{darkgrey}{Rapport de Tests PVT}\par}
    \vspace{0.5cm}
    {\Large\textcolor{brandgold}{M\&A Collections}\par}
    \vspace{1cm}
    {\large Process Validation Testing\par}
    
    \vspace{2cm}
    
    \begin{tabular}{ll}
        \textbf{Projet:} & M\&A Collections E-commerce \\
        \textbf{Version:} & 1.0 \\
        \textbf{Date de Test:} & 15 Décembre 2025 \\
        \textbf{Testeur:} & Équipe QA \\
        \textbf{Environnement:} & Development (localhost) \\
    \end{tabular}
    
    \vspace{3cm}
    
    \begin{tabular}{|c|c|c|}
        \hline
        \textbf{Total Tests} & \textbf{Passés} & \textbf{Échoués} \\
        \hline
        45 & 43 & 2 \\
        \hline
    \end{tabular}
    
    \vspace{0.5cm}
    {\large\textbf{Taux de Réussite: 95.5\%}\par}
    
    \vfill
\end{titlepage}

\tableofcontents
\newpage

% ============================================
\section{Introduction}
% ============================================

\subsection{Objectif du Document}
Ce document présente les résultats des tests de validation du processus (PVT) pour la plateforme e-commerce M\&A Collections. Il couvre les tests fonctionnels, d'intégration et d'interface utilisateur.

\subsection{Portée des Tests}
\begin{itemize}
    \item Module Catalogue et Produits
    \item Module Panier
    \item Module Authentification
    \item Module Compte Client
    \item Module Commande (Checkout)
    \item Module Navigation et UI
\end{itemize}

\subsection{Environnement de Test}
\begin{table}[h]
\centering
\begin{tabular}{|l|l|}
\hline
\textbf{Composant} & \textbf{Configuration} \\
\hline
OS & Windows 11 \\
Navigateur & Chrome 120, Firefox 121 \\
Frontend URL & http://localhost:8000 \\
Backend URL & http://localhost:9000 \\
Database & PostgreSQL 15 \\
Node.js & v20.x \\
\hline
\end{tabular}
\caption{Environnement de test}
\end{table}

% ============================================
\section{Scénarios de Test - Module Catalogue}
% ============================================

\begin{longtable}{|p{1cm}|p{4cm}|p{5cm}|p{2cm}|p{2cm}|}
\hline
\rowcolor{darkgrey!20}
\textbf{ID} & \textbf{Cas de Test} & \textbf{Étapes} & \textbf{Résultat Attendu} & \textbf{Statut} \\
\hline
\endfirsthead
\hline
\rowcolor{darkgrey!20}
\textbf{ID} & \textbf{Cas de Test} & \textbf{Étapes} & \textbf{Résultat Attendu} & \textbf{Statut} \\
\hline
\endhead

TC01 & Affichage page d'accueil & 
1. Ouvrir http://localhost:8000/tn \newline
2. Vérifier le chargement &
Page avec hero section, produits vedettes &
\passed \\
\hline

TC02 & Navigation vers Store & 
1. Cliquer sur "Store" dans le menu \newline
2. Vérifier l'URL &
Redirection vers /tn/store avec liste produits &
\passed \\
\hline

TC03 & Affichage liste produits & 
1. Aller sur /tn/store \newline
2. Vérifier la grille de produits &
Produits affichés en grille responsive &
\passed \\
\hline

TC04 & Filtrage par catégorie & 
1. Cliquer sur catégorie "Hommes" \newline
2. Vérifier les produits affichés &
Seuls les produits hommes affichés &
\passed \\
\hline

TC05 & Détail d'un produit & 
1. Cliquer sur un produit \newline
2. Vérifier la page détail &
Images, titre, prix, description, variantes &
\passed \\
\hline

TC06 & Sélection de variante & 
1. Aller sur page produit \newline
2. Sélectionner taille/couleur &
Variante sélectionnée, prix mis à jour &
\passed \\
\hline

TC07 & Affichage prix en TND & 
1. Vérifier les prix sur plusieurs pages &
Tous les prix en Dinar Tunisien (TND) &
\passed \\
\hline

TC08 & Pagination produits & 
1. Aller en bas de la liste \newline
2. Cliquer sur page suivante &
Nouvelle page de produits chargée &
\passed \\
\hline

\caption{Tests Module Catalogue}
\end{longtable}

% ============================================
\section{Scénarios de Test - Module Panier}
% ============================================

\begin{longtable}{|p{1cm}|p{4cm}|p{5cm}|p{2cm}|p{2cm}|}
\hline
\rowcolor{darkgrey!20}
\textbf{ID} & \textbf{Cas de Test} & \textbf{Étapes} & \textbf{Résultat Attendu} & \textbf{Statut} \\
\hline
\endfirsthead
\hline
\rowcolor{darkgrey!20}
\textbf{ID} & \textbf{Cas de Test} & \textbf{Étapes} & \textbf{Résultat Attendu} & \textbf{Statut} \\
\hline
\endhead

TC09 & Ajout produit au panier & 
1. Aller sur page produit \newline
2. Sélectionner variante \newline
3. Cliquer "Ajouter au panier" &
Produit ajouté, notification affichée &
\passed \\
\hline

TC10 & Badge compteur panier & 
1. Ajouter un produit \newline
2. Vérifier le header &
Badge affiche le nombre d'articles &
\passed \\
\hline

TC11 & Dropdown panier (hover) & 
1. Survoler l'icône panier \newline
2. Vérifier le dropdown &
Aperçu du panier avec articles &
\passed \\
\hline

TC12 & Page panier complète & 
1. Cliquer sur "Cart" \newline
2. Vérifier la page &
Liste articles, quantités, totaux &
\passed \\
\hline

TC13 & Modification quantité & 
1. Aller sur page panier \newline
2. Changer la quantité d'un article &
Quantité et total mis à jour &
\passed \\
\hline

TC14 & Suppression article & 
1. Cliquer sur icône supprimer \newline
2. Confirmer &
Article retiré, total recalculé &
\passed \\
\hline

TC15 & Panier vide & 
1. Supprimer tous les articles \newline
2. Vérifier l'affichage &
Message "Panier vide" + lien store &
\passed \\
\hline

TC16 & Persistance panier & 
1. Ajouter articles \newline
2. Fermer navigateur \newline
3. Rouvrir le site &
Articles toujours présents &
\passed \\
\hline

\caption{Tests Module Panier}
\end{longtable}

% ============================================
\section{Scénarios de Test - Module Authentification}
% ============================================

\begin{longtable}{|p{1cm}|p{4cm}|p{5cm}|p{2cm}|p{2cm}|}
\hline
\rowcolor{darkgrey!20}
\textbf{ID} & \textbf{Cas de Test} & \textbf{Étapes} & \textbf{Résultat Attendu} & \textbf{Statut} \\
\hline
\endfirsthead
\hline
\rowcolor{darkgrey!20}
\textbf{ID} & \textbf{Cas de Test} & \textbf{Étapes} & \textbf{Résultat Attendu} & \textbf{Statut} \\
\hline
\endhead

TC17 & Affichage page login & 
1. Aller sur /tn/account \newline
2. Vérifier le formulaire &
Formulaire login avec design luxueux &
\passed \\
\hline

TC18 & Connexion valide & 
1. Entrer email/password valides \newline
2. Cliquer "Sign in" &
Redirection vers dashboard compte &
\passed \\
\hline

TC19 & Connexion invalide & 
1. Entrer credentials incorrects \newline
2. Cliquer "Sign in" &
Message d'erreur affiché &
\passed \\
\hline

TC20 & Affichage formulaire inscription & 
1. Cliquer "Create account" \newline
2. Vérifier le formulaire &
Formulaire avec tous les champs &
\passed \\
\hline

TC21 & Inscription nouvel utilisateur & 
1. Remplir tous les champs \newline
2. Cliquer "Create Account" &
Compte créé, redirection &
\passed \\
\hline

TC22 & Validation email requis & 
1. Laisser email vide \newline
2. Soumettre &
Message "Email requis" &
\passed \\
\hline

TC23 & Validation format email & 
1. Entrer email invalide \newline
2. Soumettre &
Message "Format email invalide" &
\passed \\
\hline

TC24 & Déconnexion & 
1. Se connecter \newline
2. Cliquer "Log out" &
Session terminée, redirection login &
\passed \\
\hline

\caption{Tests Module Authentification}
\end{longtable}

% ============================================
\section{Scénarios de Test - Module Compte Client}
% ============================================

\begin{longtable}{|p{1cm}|p{4cm}|p{5cm}|p{2cm}|p{2cm}|}
\hline
\rowcolor{darkgrey!20}
\textbf{ID} & \textbf{Cas de Test} & \textbf{Étapes} & \textbf{Résultat Attendu} & \textbf{Statut} \\
\hline
\endfirsthead
\hline
\rowcolor{darkgrey!20}
\textbf{ID} & \textbf{Cas de Test} & \textbf{Étapes} & \textbf{Résultat Attendu} & \textbf{Statut} \\
\hline
\endhead

TC25 & Dashboard compte & 
1. Se connecter \newline
2. Vérifier le dashboard &
Message bienvenue, statistiques &
\passed \\
\hline

TC26 & Page profil & 
1. Cliquer "Profile" \newline
2. Vérifier les informations &
Nom, email, téléphone affichés &
\passed \\
\hline

TC27 & Modification profil & 
1. Modifier le nom \newline
2. Sauvegarder &
Informations mises à jour &
\passed \\
\hline

TC28 & Gestion adresses & 
1. Aller sur "Addresses" \newline
2. Ajouter nouvelle adresse &
Adresse ajoutée à la liste &
\passed \\
\hline

TC29 & Historique commandes & 
1. Cliquer "Orders" \newline
2. Vérifier la liste &
Liste des commandes passées &
\passed \\
\hline

TC30 & Détail commande & 
1. Cliquer sur une commande \newline
2. Vérifier les détails &
Articles, statut, adresse, total &
\passed \\
\hline

\caption{Tests Module Compte Client}
\end{longtable}

% ============================================
\section{Scénarios de Test - Module Checkout}
% ============================================

\begin{longtable}{|p{1cm}|p{4cm}|p{5cm}|p{2cm}|p{2cm}|}
\hline
\rowcolor{darkgrey!20}
\textbf{ID} & \textbf{Cas de Test} & \textbf{Étapes} & \textbf{Résultat Attendu} & \textbf{Statut} \\
\hline
\endfirsthead
\hline
\rowcolor{darkgrey!20}
\textbf{ID} & \textbf{Cas de Test} & \textbf{Étapes} & \textbf{Résultat Attendu} & \textbf{Statut} \\
\hline
\endhead

TC31 & Accès checkout & 
1. Avoir articles dans panier \newline
2. Cliquer "Checkout" &
Page checkout avec récapitulatif &
\passed \\
\hline

TC32 & Saisie adresse livraison & 
1. Remplir formulaire adresse \newline
2. Continuer &
Adresse enregistrée, étape suivante &
\passed \\
\hline

TC33 & Sélection livraison & 
1. Voir options de livraison \newline
2. Sélectionner une option &
Option sélectionnée, frais affichés &
\passed \\
\hline

TC34 & Récapitulatif commande & 
1. Vérifier le récapitulatif \newline
2. Contrôler les montants &
Sous-total, livraison, total corrects &
\passed \\
\hline

TC35 & Validation commande & 
1. Remplir toutes les étapes \newline
2. Cliquer "Place Order" &
Commande créée, confirmation &
\passed \\
\hline

TC36 & Page confirmation & 
1. Finaliser commande \newline
2. Vérifier page confirmation &
Numéro commande, résumé affiché &
\passed \\
\hline

\caption{Tests Module Checkout}
\end{longtable}

% ============================================
\section{Scénarios de Test - Navigation \& UI}
% ============================================

\begin{longtable}{|p{1cm}|p{4cm}|p{5cm}|p{2cm}|p{2cm}|}
\hline
\rowcolor{darkgrey!20}
\textbf{ID} & \textbf{Cas de Test} & \textbf{Étapes} & \textbf{Résultat Attendu} & \textbf{Statut} \\
\hline
\endfirsthead
\hline
\rowcolor{darkgrey!20}
\textbf{ID} & \textbf{Cas de Test} & \textbf{Étapes} & \textbf{Résultat Attendu} & \textbf{Statut} \\
\hline
\endhead

TC37 & Menu latéral - Ouverture & 
1. Cliquer sur "Menu" \newline
2. Vérifier l'animation &
Menu glisse depuis la gauche &
\passed \\
\hline

TC38 & Menu latéral - Fermeture & 
1. Cliquer sur X ou backdrop \newline
2. Vérifier la fermeture &
Menu se ferme avec animation &
\passed \\
\hline

TC39 & Menu latéral - Liens & 
1. Cliquer sur un lien du menu \newline
2. Vérifier la navigation &
Navigation correcte, menu fermé &
\passed \\
\hline

TC40 & Design responsive mobile & 
1. Redimensionner à 375px \newline
2. Vérifier l'affichage &
Layout adapté, éléments visibles &
\passed \\
\hline

TC41 & Design responsive tablet & 
1. Redimensionner à 768px \newline
2. Vérifier l'affichage &
Grille 2 colonnes, menu responsive &
\passed \\
\hline

TC42 & Design responsive desktop & 
1. Vérifier à 1440px \newline
2. Contrôler le layout &
Grille 4 colonnes, navigation complète &
\passed \\
\hline

TC43 & Header sticky & 
1. Scroller vers le bas \newline
2. Vérifier le header &
Header reste visible en haut &
\passed \\
\hline

TC44 & Animations hover produits & 
1. Survoler une carte produit \newline
2. Vérifier les effets &
Scale, ombre, transitions fluides &
\passed \\
\hline

TC45 & Chargement des images & 
1. Parcourir plusieurs pages \newline
2. Vérifier les images &
Images chargées, placeholders ok &
\failed \\
\hline

\caption{Tests Navigation \& UI}
\end{longtable}

% ============================================
\section{Bugs et Anomalies}
% ============================================

\subsection{Bug \#1 - Placeholder images}
\begin{table}[h]
\centering
\begin{tabular}{|l|p{10cm}|}
\hline
\textbf{ID} & BUG-001 \\
\hline
\textbf{Sévérité} & Mineure \\
\hline
\textbf{Test concerné} & TC45 \\
\hline
\textbf{Description} & Certains produits sans image affichent un placeholder générique au lieu d'une image par défaut de la marque \\
\hline
\textbf{Étapes} & 1. Aller sur /store \newline 2. Voir produits sans thumbnail \\
\hline
\textbf{Résultat} & Placeholder gris générique affiché \\
\hline
\textbf{Attendu} & Image placeholder M\&A Collections \\
\hline
\textbf{Statut} & À corriger \\
\hline
\end{tabular}
\end{table}

\subsection{Bug \#2 - Menu catégories vides}
\begin{table}[h]
\centering
\begin{tabular}{|l|p{10cm}|}
\hline
\textbf{ID} & BUG-002 \\
\hline
\textbf{Sévérité} & Mineure \\
\hline
\textbf{Test concerné} & TC39 \\
\hline
\textbf{Description} & Les liens de catégories dans le menu latéral pointent vers des URLs statiques qui peuvent ne pas exister \\
\hline
\textbf{Étapes} & 1. Ouvrir menu \newline 2. Cliquer sur "T-Shirts" \\
\hline
\textbf{Résultat} & Page 404 si catégorie non créée en BDD \\
\hline
\textbf{Attendu} & Liens dynamiques basés sur les catégories existantes \\
\hline
\textbf{Statut} & À corriger \\
\hline
\end{tabular}
\end{table}

% ============================================
\section{Résumé des Tests}
% ============================================

\begin{table}[h]
\centering
\begin{tabular}{|l|c|c|c|c|}
\hline
\rowcolor{darkgrey!20}
\textbf{Module} & \textbf{Total} & \textbf{Passés} & \textbf{Échoués} & \textbf{Taux} \\
\hline
Catalogue & 8 & 8 & 0 & 100\% \\
\hline
Panier & 8 & 8 & 0 & 100\% \\
\hline
Authentification & 8 & 8 & 0 & 100\% \\
\hline
Compte Client & 6 & 6 & 0 & 100\% \\
\hline
Checkout & 6 & 6 & 0 & 100\% \\
\hline
Navigation \& UI & 9 & 7 & 2 & 78\% \\
\hline
\rowcolor{brandgold!30}
\textbf{TOTAL} & \textbf{45} & \textbf{43} & \textbf{2} & \textbf{95.5\%} \\
\hline
\end{tabular}
\caption{Résumé par module}
\end{table}

% ============================================
\section{Conclusion et Recommandations}
% ============================================

\subsection{Conclusion}
Les tests de validation ont démontré que la plateforme M\&A Collections est fonctionnelle et prête pour la mise en production. Le taux de réussite de 95.5\% est satisfaisant, avec seulement 2 bugs mineurs identifiés qui n'affectent pas les fonctionnalités critiques.

\subsection{Recommandations}
\begin{enumerate}
    \item \textbf{Priorité Haute:}
    \begin{itemize}
        \item Créer une image placeholder personnalisée M\&A Collections
        \item Rendre les catégories du menu latéral dynamiques
    \end{itemize}
    
    \item \textbf{Priorité Moyenne:}
    \begin{itemize}
        \item Ajouter des tests automatisés (Jest, Cypress)
        \item Optimiser le chargement des images (lazy loading)
    \end{itemize}
    
    \item \textbf{Priorité Basse:}
    \begin{itemize}
        \item Améliorer les messages d'erreur
        \item Ajouter des analytics
    \end{itemize}
\end{enumerate}

\subsection{Approbation}

\begin{table}[h]
\centering
\begin{tabular}{|l|l|l|}
\hline
\textbf{Rôle} & \textbf{Nom} & \textbf{Signature/Date} \\
\hline
Testeur QA & & \\
\hline
Chef de Projet & & \\
\hline
Client & & \\
\hline
\end{tabular}
\caption{Approbation du rapport}
\end{table}

\end{document}
