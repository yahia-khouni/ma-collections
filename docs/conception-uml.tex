\documentclass[12pt,a4paper]{article}
\usepackage[utf8]{inputenc}
\usepackage[T1]{fontenc}
\usepackage[french]{babel}
\usepackage{geometry}
\usepackage{graphicx}
\usepackage{hyperref}
\usepackage{listings}
\usepackage{xcolor}
\usepackage{titlesec}
\usepackage{fancyhdr}

\geometry{margin=2.5cm}

\definecolor{brandgold}{RGB}{212,175,55}
\definecolor{darkgrey}{RGB}{17,24,39}
\definecolor{codegreen}{rgb}{0,0.6,0}
\definecolor{codegray}{rgb}{0.5,0.5,0.5}

\lstset{
    backgroundcolor=\color{gray!10},
    basicstyle=\ttfamily\small,
    breaklines=true,
    frame=single,
    language=Java,
    showstringspaces=false,
    keywordstyle=\color{blue},
    commentstyle=\color{codegreen},
}

\titleformat{\section}{\Large\bfseries\color{darkgrey}}{\thesection}{1em}{}
\titleformat{\subsection}{\large\bfseries\color{darkgrey}}{\thesubsection}{1em}{}

\pagestyle{fancy}
\fancyhf{}
\fancyhead[L]{\textcolor{brandgold}{M\&A Collections}}
\fancyhead[R]{Conception UML}
\fancyfoot[C]{\thepage}

\begin{document}

\begin{titlepage}
    \centering
    \vspace*{2cm}
    
    {\Huge\bfseries\textcolor{darkgrey}{Document de Conception}\par}
    \vspace{0.5cm}
    {\Large\textcolor{brandgold}{M\&A Collections}\par}
    \vspace{1cm}
    {\large Diagrammes UML - PlantUML\par}
    
    \vspace{2cm}
    
    \begin{tabular}{ll}
        \textbf{Projet:} & M\&A Collections E-commerce \\
        \textbf{Version:} & 1.0 \\
        \textbf{Date:} & 15 Décembre 2025 \\
    \end{tabular}
    
    \vfill
\end{titlepage}

\tableofcontents
\newpage

% ============================================
\section{Diagrammes de Cas d'Utilisation}
% ============================================

\subsection{Cas d'Utilisation Général}

\begin{lstlisting}[caption={Use Case - Système Global}]
@startuml UseCase_Global
left to right direction
skinparam packageStyle rectangle
skinparam actorStyle awesome

actor "Visiteur" as V
actor "Client" as C
actor "Administrateur" as A

rectangle "M&A Collections E-commerce" {
    ' Visiteur use cases
    usecase "Parcourir le catalogue" as UC1
    usecase "Rechercher un produit" as UC2
    usecase "Voir details produit" as UC3
    usecase "S'inscrire" as UC4
    usecase "Se connecter" as UC5
    
    ' Client use cases
    usecase "Ajouter au panier" as UC6
    usecase "Gerer le panier" as UC7
    usecase "Passer une commande" as UC8
    usecase "Consulter historique" as UC9
    usecase "Gerer son profil" as UC10
    usecase "Se deconnecter" as UC11
    
    ' Admin use cases
    usecase "Gerer les produits" as UC12
    usecase "Gerer les commandes" as UC13
    usecase "Gerer les clients" as UC14
    usecase "Voir statistiques" as UC15
}

V --> UC1
V --> UC2
V --> UC3
V --> UC4
V --> UC5

C --> UC1
C --> UC2
C --> UC3
C --> UC6
C --> UC7
C --> UC8
C --> UC9
C --> UC10
C --> UC11

A --> UC12
A --> UC13
A --> UC14
A --> UC15

C --|> V : extends

@enduml
\end{lstlisting}

\subsection{Cas d'Utilisation - Module Panier}

\begin{lstlisting}[caption={Use Case - Gestion du Panier}]
@startuml UseCase_Cart
left to right direction
skinparam actorStyle awesome

actor "Client" as C

rectangle "Module Panier" {
    usecase "Ajouter produit au panier" as UC1
    usecase "Modifier quantite" as UC2
    usecase "Supprimer article" as UC3
    usecase "Voir panier" as UC4
    usecase "Appliquer code promo" as UC5
    usecase "Calculer total" as UC6
    usecase "Vider le panier" as UC7
    usecase "Proceder au checkout" as UC8
}

C --> UC1
C --> UC2
C --> UC3
C --> UC4
C --> UC5
C --> UC7
C --> UC8

UC1 ..> UC6 : <<include>>
UC2 ..> UC6 : <<include>>
UC3 ..> UC6 : <<include>>
UC5 ..> UC6 : <<include>>
UC4 ..> UC6 : <<include>>

@enduml
\end{lstlisting}

\subsection{Cas d'Utilisation - Module Commande}

\begin{lstlisting}[caption={Use Case - Processus de Commande}]
@startuml UseCase_Checkout
left to right direction
skinparam actorStyle awesome

actor "Client" as C
actor "Systeme Paiement" as SP
actor "Systeme Email" as SE

rectangle "Module Commande (Checkout)" {
    usecase "Saisir adresse livraison" as UC1
    usecase "Choisir mode livraison" as UC2
    usecase "Choisir mode paiement" as UC3
    usecase "Valider commande" as UC4
    usecase "Traiter paiement" as UC5
    usecase "Confirmer commande" as UC6
    usecase "Envoyer confirmation" as UC7
}

C --> UC1
C --> UC2
C --> UC3
C --> UC4

UC4 ..> UC5 : <<include>>
UC5 --> SP
UC5 ..> UC6 : <<include>>
UC6 ..> UC7 : <<include>>
UC7 --> SE

@enduml
\end{lstlisting}

\subsection{Cas d'Utilisation - Module Authentification}

\begin{lstlisting}[caption={Use Case - Authentification}]
@startuml UseCase_Auth
left to right direction
skinparam actorStyle awesome

actor "Visiteur" as V
actor "Client" as C

rectangle "Module Authentification" {
    usecase "S'inscrire" as UC1
    usecase "Se connecter" as UC2
    usecase "Se deconnecter" as UC3
    usecase "Reinitialiser mot de passe" as UC4
    usecase "Modifier mot de passe" as UC5
    usecase "Valider email" as UC6
}

V --> UC1
V --> UC2
V --> UC4

C --> UC3
C --> UC5

UC1 ..> UC6 : <<include>>
UC4 ..> UC6 : <<include>>

@enduml
\end{lstlisting}

\newpage
% ============================================
\section{Diagrammes de Classes}
% ============================================

\subsection{Diagramme de Classes Principal}

\begin{lstlisting}[caption={Class Diagram - Entites Principales}]
@startuml Class_Main
skinparam classAttributeIconSize 0
skinparam classFontStyle bold

class Customer {
    - id: string
    - email: string
    - first_name: string
    - last_name: string
    - phone: string
    - password_hash: string
    - created_at: datetime
    - updated_at: datetime
    + register(): void
    + login(): boolean
    + logout(): void
    + updateProfile(): void
}

class Address {
    - id: string
    - first_name: string
    - last_name: string
    - address_1: string
    - address_2: string
    - city: string
    - postal_code: string
    - country_code: string
    - phone: string
}

class Product {
    - id: string
    - title: string
    - subtitle: string
    - description: string
    - handle: string
    - thumbnail: string
    - status: ProductStatus
    - created_at: datetime
    + getPrice(region): Money
    + isAvailable(): boolean
}

class ProductVariant {
    - id: string
    - title: string
    - sku: string
    - inventory_quantity: int
    - prices: Price[]
    + isInStock(): boolean
}

class ProductCategory {
    - id: string
    - name: string
    - handle: string
    - description: string
    - parent_category_id: string
}

class Cart {
    - id: string
    - email: string
    - region_id: string
    - items: LineItem[]
    - subtotal: int
    - total: int
    + addItem(variant, quantity): void
    + removeItem(itemId): void
    + updateQuantity(itemId, qty): void
    + getTotal(): Money
}

class LineItem {
    - id: string
    - cart_id: string
    - variant_id: string
    - quantity: int
    - unit_price: int
    - total: int
}

class Order {
    - id: string
    - display_id: int
    - customer_id: string
    - status: OrderStatus
    - fulfillment_status: string
    - payment_status: string
    - items: LineItem[]
    - shipping_address: Address
    - total: int
    - created_at: datetime
}

class Region {
    - id: string
    - name: string
    - currency_code: string
    - countries: Country[]
}

enum ProductStatus {
    DRAFT
    PUBLISHED
    REJECTED
}

enum OrderStatus {
    PENDING
    COMPLETED
    CANCELLED
    ARCHIVED
}

Customer "1" -- "*" Address : has
Customer "1" -- "*" Order : places
Customer "1" -- "0..1" Cart : has

Product "1" -- "*" ProductVariant : contains
Product "*" -- "*" ProductCategory : belongs to

Cart "1" -- "*" LineItem : contains
LineItem "*" -- "1" ProductVariant : references

Order "1" -- "*" LineItem : contains
Order "1" -- "1" Address : ships to
Order "*" -- "1" Region : belongs to

Cart "*" -- "1" Region : belongs to

@enduml
\end{lstlisting}

\subsection{Diagramme de Classes - Module Panier}

\begin{lstlisting}[caption={Class Diagram - Module Panier}]
@startuml Class_Cart
skinparam classAttributeIconSize 0

class CartService {
    + createCart(regionId): Cart
    + getCart(cartId): Cart
    + addLineItem(cartId, variantId, qty): Cart
    + updateLineItem(cartId, itemId, qty): Cart
    + removeLineItem(cartId, itemId): Cart
    + applyDiscount(cartId, code): Cart
    + calculateTotals(cart): Cart
}

class Cart {
    - id: string
    - region_id: string
    - customer_id: string
    - email: string
    - billing_address_id: string
    - shipping_address_id: string
    - items: LineItem[]
    - discounts: Discount[]
    - subtotal: int
    - discount_total: int
    - shipping_total: int
    - tax_total: int
    - total: int
}

class LineItem {
    - id: string
    - cart_id: string
    - order_id: string
    - variant_id: string
    - title: string
    - description: string
    - thumbnail: string
    - quantity: int
    - unit_price: int
    - subtotal: int
    - total: int
}

class Discount {
    - id: string
    - code: string
    - is_dynamic: boolean
    - rule: DiscountRule
    - starts_at: datetime
    - ends_at: datetime
}

class DiscountRule {
    - id: string
    - type: DiscountType
    - value: int
    - allocation: string
}

enum DiscountType {
    FIXED
    PERCENTAGE
    FREE_SHIPPING
}

CartService --> Cart : manages
Cart "1" *-- "*" LineItem : contains
Cart "*" -- "*" Discount : applies

@enduml
\end{lstlisting}

\subsection{Diagramme de Classes - Frontend Components}

\begin{lstlisting}[caption={Class Diagram - React Components}]
@startuml Class_Components
skinparam classAttributeIconSize 0

package "Layout Components" {
    class Nav {
        + regions: Region[]
        + render(): JSX
    }
    
    class SideMenu {
        - isOpen: boolean
        - regions: Region[]
        + open(): void
        + close(): void
        + render(): JSX
    }
    
    class CartDropdown {
        - cart: Cart
        - isOpen: boolean
        + render(): JSX
    }
    
    class Footer {
        + collections: Collection[]
        + categories: Category[]
        + render(): JSX
    }
}

package "Product Components" {
    class ProductCard {
        - product: Product
        - region: Region
        + render(): JSX
    }
    
    class ProductGrid {
        - products: Product[]
        + render(): JSX
    }
    
    class ProductActions {
        - product: Product
        - variant: Variant
        + addToCart(): void
        + render(): JSX
    }
}

package "Account Components" {
    class Login {
        - email: string
        - password: string
        + handleSubmit(): void
        + render(): JSX
    }
    
    class Register {
        - formData: CustomerData
        + handleSubmit(): void
        + render(): JSX
    }
    
    class AccountNav {
        - customer: Customer
        + handleLogout(): void
        + render(): JSX
    }
}

package "Checkout Components" {
    class CheckoutForm {
        - cart: Cart
        - step: CheckoutStep
        + render(): JSX
    }
    
    class Addresses {
        - addresses: Address[]
        + selectAddress(): void
        + render(): JSX
    }
    
    class Payment {
        - paymentMethods: PaymentMethod[]
        + processPayment(): void
        + render(): JSX
    }
}

Nav *-- SideMenu
Nav *-- CartDropdown
ProductGrid *-- ProductCard

@enduml
\end{lstlisting}

\newpage
% ============================================
\section{Diagramme de Séquence}
% ============================================

\subsection{Séquence - Ajout au Panier}

\begin{lstlisting}[caption={Sequence Diagram - Add to Cart}]
@startuml Sequence_AddToCart
skinparam sequenceMessageAlign center

actor Client
participant "ProductPage" as PP
participant "CartService" as CS
participant "MedusaAPI" as API
participant "Database" as DB

Client -> PP: Click "Add to Cart"
activate PP

PP -> CS: addToCart(variantId, quantity)
activate CS

CS -> API: POST /store/carts/{id}/line-items
activate API

API -> DB: Check variant stock
activate DB
DB --> API: Stock available
deactivate DB

API -> DB: Insert line item
activate DB
DB --> API: Item created
deactivate DB

API -> DB: Calculate totals
activate DB
DB --> API: Updated cart
deactivate DB

API --> CS: Cart with new item
deactivate API

CS --> PP: Updated cart state
deactivate CS

PP -> PP: Show success notification
PP -> PP: Update cart badge count

PP --> Client: Display updated UI
deactivate PP

@enduml
\end{lstlisting}

\subsection{Séquence - Processus de Commande}

\begin{lstlisting}[caption={Sequence Diagram - Checkout Process}]
@startuml Sequence_Checkout
skinparam sequenceMessageAlign center

actor Client
participant "CheckoutPage" as CP
participant "MedusaAPI" as API
participant "PaymentGateway" as PG
participant "EmailService" as ES

Client -> CP: Enter shipping address
CP -> API: PUT /store/carts/{id}
API --> CP: Cart updated

Client -> CP: Select shipping method
CP -> API: POST /store/carts/{id}/shipping-methods
API --> CP: Shipping added

Client -> CP: Enter payment info
CP -> API: POST /store/carts/{id}/payment-sessions
API --> CP: Payment session created

Client -> CP: Click "Place Order"
activate CP

CP -> API: POST /store/carts/{id}/complete
activate API

API -> PG: Process payment
activate PG
PG --> API: Payment successful
deactivate PG

API -> API: Create Order
API -> ES: Send confirmation email
activate ES
ES --> API: Email sent
deactivate ES

API --> CP: Order completed
deactivate API

CP --> Client: Show order confirmation
deactivate CP

@enduml
\end{lstlisting}

% ============================================
\section{Instructions de Génération}
% ============================================

Pour générer les diagrammes à partir du code PlantUML:

\begin{enumerate}
    \item Installer PlantUML: \texttt{npm install -g node-plantuml}
    \item Ou utiliser le serveur en ligne: \url{https://www.plantuml.com/plantuml}
    \item Copier le code entre \texttt{@startuml} et \texttt{@enduml}
    \item Générer l'image (PNG, SVG, ou PDF)
\end{enumerate}

\end{document}
